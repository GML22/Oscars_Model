\newpage
\chapter*{Rozdział 5 \\ \vspace{1cm} Wnioski końcowe}
 \addcontentsline{toc}{chapter}{5. Wnioski końcowe}
 
Przeprowadzona estymacja miała na celu oszacowanie czynników wpływających na prawdopodobieństwo zdobycia Oscara przez film choć w jednej kategorii. Badanie zawarte w niniejszej pracy miało stanowić uogólnienie przeprowadzonych dotychczas badań dotyczących prawdopodobieństw zdobycia nagrody Amerykańskiej Akademii Sztuki i Wiedzy Filmowej w konkretnych kategoriach. 

Wyniki przeprowadzonej w niniejszej pracy analizy wskazują, iż na prawdopodobieństwo zdobycia Oscara wpływa ograniczona liczba czynników, z których najważniejszymi są te dotyczące innych nagród i nominacji dla danego filmu. Wnioski te pokrywają się z wnioskami zawartymi w literaturze dotyczącej badanego zagadnienia. Głównymi czynnikami wpływającymi na prawdopodobieństwo zdobycia najważniejszej w świeci filmowym statuetki są: gatunku, kraju produkcji, przychodów z kas kinowych, liczby nominacji do Oscara, liczby zdobytych Złotych Globów, liczby zdobytych nagród Brytyjskiej Akademii Sztuk Filmowych i Telewizyjnych i faktu czy w filmie występuje wiodący wątek miłosny. 

Spośród zmiennych nieistotnych w modelu najbardziej dziwi nieistotność zmiennej zero-jedynkowej ekranizacja, która reprezentowała filmy będące adaptacją powieści, biografii, sztuki lub artykułu. Większość najwybitniejszych dzieł filmowych powstało właśnie w oparciu o scenariusz jakiegoś wybitnego dzieła literackiego, stąd na wstępie pracy została zawarta hipoteza, iż zmienna ta będzie istotnie wpływać na badane zjawisko. Dziwi również, iż z dziewięciu głównych gatunków filmowych tylko filmy animowane okazały się istotnie wpływać na prawdopodobieństwo zdobycia Oscara. Hipotezy o tym, iż gatunki dramat i komedia powinny mieć wpływ na szacowane zjawisko należy odrzucić jako nieistotne. Są to wnioski stojące w opozycji do badań Dawida Kaplana i Andrew B. Bernarda omówionych na wstępie pracy.

Zarówno badanie zawarte w niniejszej pracy jak i badania omówione w literaturze wskazują wyraźnie, iż czynnikami najsilniej wpływającymi na prawdopodobieństwo zdobycia Oscara są: liczba nominacji, liczba zdobytych Złotych Globów oraz liczba zdobytych nagród BAFTA. Zmienne te były istotne we wszystkich oszacowanych modelach na niemal każdym poziomie istotności. Również ich krańcowy wpływ na modelowane prawdopodobieństwo znacznie przewyższał wpływ pozostałych zmiennych.

Nie potwierdziły się hipotezy badawcze dotyczące kraju produkcji filmu oraz wiodącego wątku miłosnego. Zakładały one, iż filmy powstałe wyłącznie w Stanach Zjednoczonych będą lepiej oceniane przez członków Akademii niż pozostałe filmy oraz występowanie wiodącego wątku miłosnego miało nie wpływać na prawdopodobieństwo zdobycia Oscara. Okazało się, że filmy tworzone poza USA lub przy współpracy USA mają większą szansę na statuetkę niż filmy czysto amerykańskie. Również wiodący wątek miłosny był zmienną istotną w modelu wpływając znacznie na modelowane zjawisko. 

Jedyną zmienną ekonomiczną, która okazała się być istotną w estymowanych modelach ekonometrycznych była zmienna przychody z kas kinowych. Jej wpływ okazał się jednak marginalny - efekty cząstkowe pokazały, iż dopiero wzrost przychodów liczony w setkach milionów dolarów wpływa znacząco na szacowane prawdopodobieństwo. W związku z takimi wynikami regresji należy stwierdzić, iż czynniki ekonomiczne nie wpływają, lub wpływają w zupełnie minimalnym stopniu na prawdopodobieństwo zdobycia Oscara. Umieszczenie czynników ekonomicznych w modelu pośrednio na przekór literaturze nie opłaciło się, model zweryfikował tę decyzję na korzyść literatury.

Wydaje się, iż zaprezentowane w niniejszej pracy badanie przyczynia się do lepszego zrozumienia czynników wpływających na prawdopodobieństwo zdobycia Oscara, wskazuje które czynniki są najbardziej istotne i mają największy wpływ na decyzje podejmowane przez członków Amerykańskiej Akademii Sztuki i Wiedzy Filmowej.