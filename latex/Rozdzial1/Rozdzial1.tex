\newpage
\chapter*{Rozdział 1 \\ \vspace{1cm} Podstawy teoretyczne modelu}
 \addcontentsline{toc}{chapter}{1. Podstawy teoretyczne modelu}

W rozdziale tym opisane zostaną wcześniejsze najbardziej znane badania ekonometryczne poruszające zagadnienie modelowania prawdopodobieństwa zdobycia nagrody Amerykańskiej Akademii Sztuki i Wiedzy Filmowej. Szczególny nacisk zostanie położony na wskazanie celów poszczególnych badań, hipotez badawczych, metod estymacji, sposobu doboru danych oraz ostatecznych wniosków. W dalszej części rozdziału zaprezentowane zostaną hipotezy badawcze niniejszej pracy będące konsekwencją analizy wcześniej przytoczonej literatury oraz wyrosłe z wiedzy teoretycznej. Każda hipoteza zostanie szczegółowo opisana i przeanalizowana, tak by móc ją poddać stosownej weryfikacji w finalnej części pracy. 
 
\phantomsection	 			% do hiperlinków dla sekcji w spisie treści
\section*{1.1 Literatura dotycząca problemu}
\addcontentsline{toc}{section}{1.1 Literatura poruszająca problem}

Iain Pardoe i Dean K. Simonton w artykule \textit{Applying discrete choice models to predict Academy Award winners} \cite{pardoe08} opisali swoje badania dotyczące prawdopodobieństwa otrzymania Oscara w czterech kategoriach: najlepszy film, najlepszy reżyser, najlepszy aktor pierwszoplanowy i najlepsza aktorka pierwszoplanowa. W swoich badaniach wykorzystali oni dane portalu \textit{us.imdb.com} dotyczące filmów nominowanych do Oscara w latach 1938-2006. Celem ich badań było przewidzenie zwycięzców Oscarów w poszczególnych kategoriach w latach 1938-2006 wykorzystując informacje dotyczące nominowanych filmów/osób dostępne przed ogłoszeniem werdyktu Amerykańskiej Akademii Filmowej. W tym celu posłużono się modelem \textbf{ mixed logit}, który jest rozszerzeniem klasycznego logitu wielomianowego. Głównymi hipotezami badawczymi w artykule I. Pardoe'a i D.K. Simonton'a były: na prawdopodobieństwo zdobycia Oscara w czterech głównych kategoriach wpływa liczba nominacji i liczba wygranych Złotych Globów w tożsamych kategoriach, na szansę zdobycia Oscara dla najlepszego reżysera ma wpływ to czy był on już nim nagradzany i czy dostał nagrodę \textit{Directors Guild of America Award} \footnote{ Directors Guild of America Award to nagroda przyznawana przez Amerykańską Gildię Reżyserów dla najlepszego reżysera danego roku, jej wręczenie ma miejsce około miesiąc przed galą rozdania Oscarów}, większe prawdopodobieństwo otrzymania Oscara ma aktor/aktorka, która była już wcześniej nominowana/wygrała tę nagrodę. Autorzy w swym badaniu wykorzystali następujące zmienne niezależne: całkowita liczba nominacji, nominacja dla reżysera, nominacja dla najlepszego filmu, wygrany Złoty Glob w kategorii dramat, wygrany Złoty Glob w kategorii komedia lub musical, wygrana przez reżysera nagroda Gildii, poprzednie oscarowe wygrane i nominacje dla ludzi zaangażowanych w film. Główne wnioski wypływające z omawianego w tym paragrafie artykułu wskazują na fakt, iż czynniki wpływające na zdobycie Oscara w poszczególnych kategoriach zmieniają się w  czasie. Co nie zmieniło się od 1938 roku to pozytywny wpływ  liczby nominacji (kształtował się on na poziomie około 0,5\%) oraz zdobycia Złotych Globów w tożsamych kategoriach (szacowany wpływ - między 2, a 4\%). Badanie przeprowadzone w przytoczonym w tym rozdziale artykule pozwoliło z zadowalającym wynikiem zakwalifikować zwycięzców w kategoriach: najlepszy aktor(77\% dobrych trafień), najlepszy film (70\%), najlepsza aktorka (77\%) oraz najlepszy film (93\%). 

Andrew B. Bernard w swoim artykule \textit{An Index of Oscar-Worthiness: Predicting the Academy Award for Best Picture} \cite{bernard05} starał się zbadać czynniki, które mogą wpłynąć na prawdopodobieństwo zdobycia Oscara w najbardziej prestiżowej kategorii, czyli najlepszy film. Do tego celu zebrał on informacje o filmach nominowanych do Oscara z lat 1984-2003. Zmienne w badaniu podzielił na 2 kategorie: charakteryzujące film i mierzące liczbę wyróżnień filmu. Oprócz zmiennych przytoczonych już w artykule omawianym w poprzednim paragrafie autor wykorzystał takie zmienne jak: pochodzenie głównego bohatera, niepełnosprawność głównego bohatera, nieszczęśliwa miłość, ekranizacja, czy wojenny charakter filmu. Po wstępnym oszacowaniu modelu \textbf{probit} okazało się, że zmiennymi istotnymi w badaniu są: całkowita liczba nominacji do Oscara, całkowita liczba wygranych Złotych Globów, wygranie Złotego Globa w kategorii najlepszy film, pochodzenie głównego bohatera, zero-jedynkowa zmienna wskazująca czy główny bohater jest geniuszem, oraz zero-jedynkowa zmienna wskazująca czy film jest komedią czy nie. Najlepiej przewidującym zdobycie Oscara (18 na 20 poprawnie zakwalifikowanych zwycięzców) okazał się model z trzema zmiennymi niezależnymi: całkowitą liczbą nominacji do Oscara (każda kolejna nominacja zwiększa prawdopodobieństwo otrzymania Oscara o 4,5\%),całkowitą liczbą wygranych Złotych Globów (każdy kolejny Złoty Glob zwiększa prawdopodobieństwo otrzymania Oscara o 10,2\%) oraz zero-jedynkową zmienną wskazującą czy film jest komedią czy nie (fakt bycia komedią w 100\% przewidywał porażkę). 

Ostatnim artykułem, który zostanie omówiony w tej części rozdziału jest artykuł Davida Kaplana \textit{And the Oscar Goes to…A Logistic Regression Model for Predicting Academy Award Results} \cite{kaplan06}, który podobnie jak praca Andrew B. Bernard'a skupia się na modelowaniu prawdopodobieństwa otrzymania Oscara w kategorii najlepszy film. Kaplan do swoich badań wykorzystał dwie bazy danych dotyczące filmów z lat 1966-2006: \textit{Videohound’s Golden Movie Retriever} i \textit{Internet Movie Database (IMDB)}. Jako model badawczy przyjął model \textbf{logit} analizując informacje dotyczące 200 filmów, dla których utworzył on 22 zmienne niezależne. Tylko 9 zmiennych z pierwotnych 22 okazało się istotne w badaniu i mogło służyć do właściwego wnioskowania, były nimi (w nawiasach ilorazy szans poszczególnych zmiennych): zero-jedynkowa zmienna wskazująca czy film jest biografią (0,0167), zero-jedynkowa zmienna wskazująca czy film miał najwięcej nominacji w stawce (15,7824), zero-jedynkowa zmienna wskazująca czy film wygrał Złotego Globa w kategorii komedia/musical (7,1135), zero-jedynkowa zmienna wskazująca czy film wygrał Złotego Globa w kategorii dramat (36,8664), zero-jedynkowa zmienna wskazująca czy reżyser dostał Złotego Globa (0,1816), zero-jedynkowa zmienna wskazująca czy film był wybitny i był biografią (645,6128), liczba poprzednich nominacji reżysera do Oscara (0,3599), zero-jedynkowa zmienna wskazująca czy reżyser filmu dostał za ten film nagrodę Amerykańskiej Gildii Reżyserów (307,3847) oraz liczba Oscarów dla najlepszego aktora/aktorki, które obsada zdobyła przed nakręceniem danego filmu (0,3046). Pseudo R$^2$ modelu, którym posługiwał się David Kaplan wyniosło około 0,5. Wyniki badania przeprowadzonego przez autora artykułu omawianego w tym podrozdziale potwierdzają główne wnioski płynące z pozostałych artykułów - najbardziej istotnym czynnikiem wpływającym na prawdopodobieństwo otrzymania Oscara są poprzednie nagrody, które uzyskał film, lub osoby tworzące film. 

Mankamentem większość badań poruszających zagadnienie czynników wpływających na szanse oskarowe filmu jest fakt, iż koncentrują się one na konkretnej kategorii nagrody nie starając się oszacować modelu ogólnego, czyli prawdopodobieństwa otrzymania Oscara w jakiejkolwiek kategorii. W badaniach pomijane są również czynniki ekonomiczne takie jak budżet filmu, przychody z kas kinowych,czy zyskowność filmu. Najczęściej próbką testową dla badaczy są filmy nominowane do nagrody Akademii, a nie losowa próbka filmów, co zdecydowanie zawęża pole wnioskowania i zuboża analizy. W niniejszej pracy postaram się wypełnić tą niewątpliwą lukę badawczą tworząc model dla losowej próbki filmów, będący uogólnieniem innych badań - skupiając się na prawdopodobieństwie otrzymania Oscara bez podziału na kategorię oraz umieszczając w modelu zmienne niezależne o charakterze ekonomicznym.
 
\phantomsection				% do hiperlinków dla sekcji w spisie treści
\section*{1.2 Hipotezy badawcze}
\addcontentsline{toc}{section}{1.2 Hipotezy badawcze}

Analizując literaturę przytoczoną w poprzedniej części tego rozdziału można dojść do wniosku, iż istnieje co najmniej kilka czynników istotnie wpływających na prawdopodobieństwo otrzymania Oscara w poszczególnych kategoriach nagrody. Większość z nich wpływa dodatnio na badane zjawisko. Celem niniejszej pracy będzie weryfikacja czy wnioski wynikające z literatury można uogólnić na wszystkie kategorie oskarowe jako całość oraz znalezienie innych istotnych czynników wpływających na modelowane prawdopodobieństwo. Wiąże się to z odpowiedzią na następujące pytania badawcze:

\begin{itemize}
    \item Czy gatunek wiodący filmu ma wpływ na jego szanse oskarowe? \vspace{0.2cm}\\
Hipoteza: Literatura i analiza danych historycznych wskazuje, iż niektóre gatunki filmowe są preferowane przez członków Akademii (np. dramat), inne zaś prawie nigdy nie zdobywają statuetki (komedia). Co skłania do postawienia hipotezy, iż taki wpływ  istnieje.
  \item Czy wielkość budżetu może wpłynąć na prawdopodobieństwo zdobycia najważniejszej statuetki w świecie filmowym? \vspace{0.2cm}\\
Hipoteza: Ze względu na charakter niektórych kategorii (np. najlepszy krótkometrażowy film animowany) budżet nie powinien istotnie wpływać na modelowane prawdopodobieństwo. Na fakt ten wskazuje również Simonton D. K. w swoim artykule \textit{Cinematic creativity and production budgets: does money make the movie?}\cite{simonton05}.
  \item Czy film będący ekranizacją ma większe szanse na Oscara? \vspace{0.2cm}\\
Hipoteza: Filmy uhonorowane największą liczbą statuetek to zazwyczaj filmy będące adaptacją jakiejś znanej powieści, czy sztuki. Stąd też hipoteza, iż czynnik ten powinien wpływać na szacowane zjawisko.  
  \item Czy przychody z kas kinowych lub inne mierniki sukcesu ekonomicznego filmu zwiększają prawdopodobieństwo zdobycia Oscara?  \vspace{0.2cm}\\
Hipoteza: Przychody z kas kinowych są pewnym rodzajem głosowania ludzi "przez portfele". Im wyższe przychody z kas do momentu głosowania przez członków Akademii tym większe szanse na zwycięstwo w choć jednej kategorii powinien mieć film.
  \item Czy amerykańskie filmy mają większe szanse oskarowe niż filmy spoza ameryki?  \vspace{0.2cm}\\
Hipoteza: Ze względu na miejsce odbywania się gali i narodowość osób głosujących filmy ze Stanów Zjednoczonych powinny mieć większe szanse w walce o Oscara niż filmy z poza USA.
  \item Jak na prawdopodobieństwo zdobycia Oscara wpływa liczba zdobytych nominacji? \vspace{0.2cm}\\
Hipoteza: Zarówno logika, jak i literatura wskazują wyraźnie, iż im więcej film ma nominacji tym większą ma szansę na zdobycie choć jednej statuetki.
  \item  Czy liczba zdobytych statuetek w innych konkursach filmowych takich jak Złote Globy czy BAFTA wpływa na prawdopodobieństwo zdobycia najważniejszej z nagród? \vspace{0.2cm}\\
Hipoteza: Większość przeprowadzonych w tym obszarze badań wskazuje , iż taka zależność istnieje i jest silna, szczególnie w przypadku Złotych Globów, które są przyznawane około miesiąc przed Oscarami.
  \item Czy wątek miłosny w filmie może zwiększyć jego szanse oskarowe? \vspace{0.2cm}\\
Hipoteza: Wydaje się, iż nie powinien to być czynnik istotny w przypadku tak prestiżowych i profesjonalnych nagród filmowych jakimi są Oscary.
  \item Czy czas trwania filmu może wpłynąć na prawdopodobieństwo zdobycia przez niego statuetki?  \vspace{0.2cm}\\
Hipoteza: Czas trwania filmu jest typowo technicznym parametrem, który nie powinien wpływać na jego ocenę.
\end{itemize}
