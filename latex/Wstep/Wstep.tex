%*********************************WSTĘP****************************************

\newpage
\chapter*{Wstęp}
 \addcontentsline{toc}{chapter}{Wstęp}
 
Każdego roku od ponad 80 lat Amerykańska Akademia Sztuki i Wiedzy Filmowej (w dalszej części pracy zwana Akademią) przyznaje najbardziej prestiżowe w świecie filmu nagrody - Oscary. Początkowo statuetki przyznawane były w 13 kategoriach, obecnie liczba kategorii wzrosła do 24. Są nimi między innymi: najlepszy film, najlepszy reżyser, najlepszy scenariusz adaptowany, najlepszy scenariusz oryginalny, najlepszy aktor, najlepsza aktorka, . Na początku każdego roku blisko 6 tysięcy członków Akademii wybiera w tajnym głosowaniu po pięć filmów w każdej kategorii, aby nominować je do nagrody. W kolejnej turze głosowania spośród nominowanych wyłaniani są zwycięzcy poszczególnych kategorii, a ich ogłoszenie ma miejsce na oficjalnej gali. Gala rozdania Oscarów jest jednym z najważniejszych wydarzeń kulturalnych nie tylko w Stanach Zjednoczonych, ale i na całym świecie. Z zapartym tchem co roku śledzą ją setki milionów ludzi, informują o niej praktycznie wszystkie serwisy informacyjne świata. 

Kwestia tego kto otrzyma Oscara od wielu lat budzi duże zainteresowanie zarówno mediów, jak i ogromnej rzeszy ludzi. Jest to temat wielu dyskusji w czasie tygodni poprzedzających galę, pojawiają się liczne teorie na ten temat, a nawet przyjmowane są zakłady bukmacherskie. Mimo to rozstrzygnięcia w poszczególnych kategoriach są często dla opinii publicznej ogromną niespodzianką. Stąd też wynika potrzeba oszacowania czynników, które mogą wpływać na prawdopodobieństwo otrzymania Oscara. Celem niniejszej pracy będzie więc skonstruowanie i przeanalizowanie odpowiedniego modelu ekonometrycznego oraz wyznaczenie zmiennych, które w sposób istotny mogą wpływać na szanse oskarowe filmu. Jest to zagadnienie rzadko pojawiające się w literaturze, nie powstało zbyt wiele publikacji na ten temat, z tego też powodu wydaje się bardzo interesujące pod kątem badawczym.

Na prawdopodobieństwo zdobycia Oscara wpływa niewątpliwie mnóstwo niemierzalnych czynników takich jak gra aktorska, klimat filmu czy ciekawość scenariusza. Istnieje jednak szereg czynników mierzalnych, które mogą zostać użyte w badaniu. Można je podzielić na trzy główne kategorie: zmienne ekonomiczne (np. budżet czy przychody filmu), zmienne charakteryzujące film (np. czas trwania, gatunek) i zmienne weryfikujące jakość filmu (liczba nagród/nominacji w innych konkursach). Wpływ tych trzech typów zmiennych na prawdopodobieństwo otrzymania Oscara będzie podlegał badaniu ekonometrycznemu przedstawionemu w niniejszej pracy. Naturalnym w tej sytuacji wydaje się posłużenie się jednym z modeli wykorzystującym binarną zmienną zależną. Jako sukces (oznaczymy przez 1) przyjmiemy zdobycie choćby jednej statuetki i porażkę (oznaczymy przez 0) zdefiniujemy jako nie zdobycie żadnej. W poniższej pracy przeanalizuję i porównam trzy modele binarnej zmiennej zależnej: Liniowy Model Prawdopodobieństwa, model Logit i model Probit. Z modeli tych wybiorę ten, który najlepiej będzie reprezentował dane.
